\section{Conclusion}

In this paper, we perform an comparative study to investigate
how tuning can improve the baseline method compared with
the state-of-art deep learning method, CNN, for predicting
knowledge units relatedness on Stack Overflow. Our experimental
results show that:

\bi
\item Tuning improves the performance of baseline methods. 
At least for WordEmbedding+SVM(baseline in \cite{xu2016predicting}) method, if not better,
it performs as well as the proposed CNN method in \cite{xu2016predicting}.
\item The baseline method with parameter tuning runs much faster than complicated deep learning.
In this study, tuning SVM runs $80X$ faster than CNN method.
\ei


Our findings have important implications for the software engineering community.
Although deep learning methods have widely and successfully applied in other community
to address problems, like pattern recognition, machine translation and recommendation
systems, it's still unclear which type of SE task is best fit for deep learning. There're lots
of possibilities in SE field to explore the application of deep learning. However, as pointed out
at the beginning of this paper, deep learning is so resource-consuming technique
and SE analytics task is usually light-weight,  {\it the  tradeoff between performance and cost should never
be ignored when applying deep learning}. The effort spent on deep learning might not be worthy the performance
improved by deep learning. As shown in our experimental results, our tuning SVM gets the similar,
 or even better, performance to deep learning while $80X$ faster.
Therefore, it should not happen any more that simply apply deep learning on SE tasks
without reporting and analyzing the resource consumption as well as the tradeoff.  

Based on the results of this study, we recommend that before applying 
deep learning method on SE tasks, consider simple techniques to improve the baseline method.
Simple methods do matter. For example, tuning parameters dramatically improve the 
performance of SVM method.  Also, Grid search should
never be used as {\it de facto} tuning method according to \cite{fu2016differential,bergstra2012random}. 
Xu et al failed to find the best tunings by using grid search for
the baseline method might confirm this point.

As to the future work, we will explore more simple techniques to solve SE tasks and also
investigate how deep learning techniques could be applied effectively in software engineering
field. 




