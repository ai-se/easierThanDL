\section{Related Work}


\subsection{Deep Learning in SE}
With the vast amounts of computational power and data, deep learning
 has been proven to be a very powerful method by researchers in many fields,
  like computer vision and natural language processing\cite{krizhevsky2012imagenet,mikolov2013distributed,sutskever2014sequence}. 
Recently, it also has attracted  attentions from researchers and practitioners in software
 community\cite{wang2016automatically, gu2016deep, xu2016predicting,white2016deep,white2015toward,lam2015combining,choetkiertikul2016deep}.
 These researchers applied  deep learning techniques to solve various problems, including 
 defect prediction, bug localization, clone code detection, API recommendation, 
 effort estimation and linkable knowledge prediction.
 
  By carefully reading, these work can be
 divided into the following two categories:
 
\begin{itemize}
\item treat deep learning as a feature extractor, and then apply other regular machine learning to do further job.
\item apply deep learning directly to solve the problems.
\end{itemize}

Lam et al.~\cite{lam2015combining}  proposed an approach to apply deep neural network
 in combination with rVSM to automatically locate the potential buggy files for a given
 bug report. By comparing it to the baseline methods(learn-to-rank\cite{ye2014learning}, 
 BugLocator~\cite{zhou2012should}), authors reported $8-20.8\%$  and $2.7-20.7\%$ 
 higher top-1 accuracy than baseline methods, respectively. The training time for deep neural
 network was reported from 70 to 122 minutes on a computer with 32 cores CPU,
 126 GB RAM machine. However,
 no such time information reported about baseline methods.
 
 Wang et al.~\cite{wang2016automatically} applied deep belief network to automatically
 learn semantic features from token vectors extracted from the studied program. After
 that, Naive Bayes and Logistic Regression methods are used to evaluate the effectiveness
 of such feature generation method as well as PROMISE and AST features. In terms of
 running time, Wang et al. only reported time for generating semantics features with deep belief network, which
 ranged from 8 seconds to 32 seconds. However, the time for training and tuning deep belief network is
 missing. Furthermore, to evaluate the effectiveness of deep belief network in terms of time, 
 it would be favorable to include all the time spent on feature extraction, including
 paring source code, token generation.
 
 Choetkiertikul et al.~\cite{choetkiertikul2016deep} proposed to apply deep learning techniques
 to solve effort estimation problem, where they used long short-term memory(LSTM) to learn
 feature vectors from the title, description and comments associated with an issue report and
 regular machine learning techniques applied afterwards. LSTM was reported to have a 
 significant improvement over the baseline method bag-of-word. No further information regarding
 runtime was reported for both methods.
 
 White et al.~\cite{white2015toward, white2016deep} applied
 recurrent neural networks, one type of  deep learning techniques, 
 to address code clone detection and code suggestion. As they reported,
 the average training time for 8 projects were ranging from 34 seconds
  to 2977 seconds for each epoch on a two 3.3 GHz
 CPUs computer and each project required at least 30 epochs~\cite{white2016deep}.
 For the {\it JDK} project in their experiment, it would take 25 hours 
 on the same computer to train the models before getting prediction.
 For the time cost for code suggestions, authors didn't mention any related information~\cite{white2015toward}.

Gu et al.~\cite{gu2016deep} proposed  a recurrent neural network(RNN)
 based method, D{\scriptsize EEP}API, to generate API usage sequences for a given natural language query. 
 Compared with the baseline method {\it SWIM}~\cite{raghothaman2016swim} and 
 {\it Lucene + UP-Miner}~\cite{wang2013mining},  D{\scriptsize EEP}API has improved the performance a lot.
 However, one can't ignore the fact that such model was trained with a Nivdia K20 GPU for about 240 hours.
 
 Xu et al.~\cite{xu2016predicting} utilized neural language model and  
 convolutional neural network(CNN) to  learn word-level and document-level features to
 predict semantically linkable knowledge units in Stack Overflow. 
 In terms of performance metrics, like precision, recall and F1-score,
 CNN method was evaluated much better than the baseline method SVM. 
 However, the time cost for training CNN is not ignorable as it required
 14 hours to achieve the loss convergence on a 2.5GHz PC with 16 GB RAM.
 

\subsection{Parameter Tuning in SE}



