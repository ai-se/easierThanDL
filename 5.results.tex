
\begin{table}[!htp]
\centering
\caption{Comparison of our baseline method with Xu et al's }
\resizebox{0.48\textwidth}{!}{
\begin{tabular} {@{}l l  c c  c c c@{}}
\hline
   \multirow{2}{*}{Metrics} &  \multirow{2}{*}{Methods} &  \multirow{2}{*}{Duplicate} &  
   \multirow{2}{*}{\begin{tabular}[c]{@{}c@{}}Direct \\ Link\end{tabular}} &
   \multirow{2}{*}{\begin{tabular}[c]{@{}c@{}}Indirect \\ Link\end{tabular}} & 
   \multirow{2}{*}{Isolated} &  \multirow{2}{*}{Overall} \\ \\ \hline
   \multirow{2}{*}{Precision}
    & Our SVM &\textbf{0.736} &0.521 & \textbf{0.793} &0.606& \textbf{0.664} \\
   & Xu's SVM &0.611 &0.560 &\textbf{0.787}&\textbf{0.676}&0.658 \\ \hline
   \multirow{2}{*}{Recall} 
   & Our SVM & 0.550& \textbf{0.502} & 0.970 & \textbf{0.645}  & 0.667 \\ 
   & Xu's SVM  & \textbf{0.725} &0.433    &\textbf{0.980}  & 0.538 &\textbf{0.669}  \\ \hline
   \multirow{2}{*}{F1-score}
   & Our SVM & 0.629 &  \textbf{0.511} &0.873  &  \textbf{0.625}&\textbf{0.660} \\ 
   & Xu's SVM & \textbf{0.663} &  0.488  & 0.873 &  0.600 &0.656 \\ \hline
   \multirow{2}{*}{Accuracy} 
   &  Our SVM&0.550  &  0.402 & 0.970 & 0.645 &0.667 \\
   & Xu's SVM & - &  -  &- &  - &\textbf{0.669} \\ \hline
 \end{tabular}}
\label{tab:baseline}
\end{table}

\section{Results}
In this section, we present our experimental results. Since we use the same data set provide by Xu
et al.\cite{xu2016predicting} and conducted our experiment in the same procedure and metics. 
Therefore, we used the results reported in the work by Xu et al.\cite{xu2016predicting} for performance
comparison. We compare the performance of tuned SVM with the state-of-art CNN method and answer
our research questions proposed in Section~\ref{RQ}.

%%%% baseline comparision %%%%%%
%\begin{table}[!htp]
%%\renewcommand{\baselinestretch}{0.35}
%%\scriptsize
%\centering
%  \caption{Comparison of our baseline method with Xu et al's }
%\resizebox{0.47\textwidth}{!}{
%  \begin{tabular} {@{}r|c c|c c|c c|c c@{}}
%   & \multicolumn{2}{c|}{Precision}  & \multicolumn{2}{c|}{Recall} &  \multicolumn{2}{c|}{F1-score} & \multicolumn{2}{c}{Accuracy}\\
%    \cline{2-9}
%    & Ours  &  \cite{xu2016predicting} & Ours  &  \cite{xu2016predicting}  &  Ours  &\cite{xu2016predicting} & Ours  &\cite{xu2016predicting}\\
%    \hline
%    Duplicate & 0.736  &0.611 & 0.550&0.725  &0.629&  0.663  &0.550&  -\\
%    Direct Link & 0.521 &0.560 & 0.502& 0.433  &0.511&  0.488& 0.402&  - \\
%    Indirect Link & 0.793  &0.787& 0.970&0.980  &0.873& 0.873  &0.970&  -\\
%    Isolated & 0.606  &0.676& 0.645 &0.538   &0.625&  0.600 & 0.645&  -\\
%    Overall & 0.664  &0.658&0.667&0.669   &0.660&  0.656 & 0.667&  0.669\\
%  \end{tabular}}
%\label{tab:baseline}
%\end{table}

\textbf{RQ1: Can we reproduce Xu et al.'s baseline results (wordembedding+SVM)?}



To answer this question, we strictly follow Xu et al.'s procedure\cite{xu2016predicting}. We 
use the SVM from scikit-learn with $\gamma = 1/200$ and $kernel= ``rbf''$. After that,
the same training and testing knowledge unit pairs are applied.

 \tab{baseline} presents the performance comparison between our baseline with
 Xu et al. in terms of accuracy, precision, recall and F1-score. As we can see, 
 when predicting these four different types of relatedness between knowledge unit pairs,
 our WordEmbedding+SVM method has  very  similar performance scores  to Xu et al.'s
, with the difference less than $0.1$.  Except for {\it Duplicate} type, where our baseline 
has a higher {\it precision} (i.e., $0.736$ v.s. $0.611$) but a lower {\it recall} (i.e., $0.550$ v.s.$0.725$).
However, our read of the averaged values of {\it accuracy},{\it precision},{\it recall}
and {\it F1-score} across four classes is that both methods have
a very small difference.

This validation  is  very important to our work since, without the original tool released by Xu et al,
we want to make sure that our reimplementation of their baseline method (WordEmbedding + SVM)does not have much difference
than theirs, which make the following study more sensible.

\begin{lesson}
Overall, our reimplementation of WordEmbedding + SVM
has very closed performance in all the evaluated metrics 
 compared to the baseline method reported in \cite{xu2016predicting}.
Therefore, our reimplementation can be treated as the baseline method in the following
experiment.
\end{lesson}


\begin{table}[!htp]
\centering
\caption{Comparison of Tuned SVM with Xu's CNN method. }
\resizebox{0.48\textwidth}{!}{
\begin{tabular} {@{}l l  c c  c c c@{}}
\hline
   \multirow{2}{*}{Metrics} &  \multirow{2}{*}{Methods} &  \multirow{2}{*}{Duplicate} &  
   \multirow{2}{*}{\begin{tabular}[c]{@{}c@{}}Direct \\ Link\end{tabular}} &
   \multirow{2}{*}{\begin{tabular}[c]{@{}c@{}}Indirect \\ Link\end{tabular}} & 
   \multirow{2}{*}{Isolated} &  \multirow{2}{*}{Overall} \\ \\ \hline
  \multirow{3}{*}{Precision} 
 % & Our SVM &0.736 &0.521 & 0.793 &0.606& 0.664 \\
   & Xu's SVM&0.611 &0.560 &0.787&0.676&0.658 \\ 
   & Xu's CNN&\textbf{0.898} & 0.758&0.840 &0.890 &0.847 \\
   & Tuned SVM&0.885 & \textbf{0.851}&\textbf{0.944} &\textbf{0.903} &\textbf{0.896}\\ \hline
   \multirow{3}{*}{Recall} 
   %& Our SVM & 0.550& 0.502 & 0.970 & 0.645  & 0.667 \\ 
   & Xu's SVM& 0.725 &0.433    &0.980  & 0.538 &0.669  \\
   & Xu's CNN& \textbf{0.898}&\textbf{0.903}    &0.773  & 0.793 &0.842  \\ 
   & Tuned SVM& 0.860 &0.828    &\textbf{0.995}  & \textbf{0.905} &\textbf{0.897}  \\  \hline
   \multirow{3}{*}{F1-score}
   %&Our SVM & 0.629 &  0.511 &0.873  &  0.625&0.660 \\ 
   & Xu's SVM& 0.663 &  0.488  & 0.873 &  0.600 &0.656 \\ 
   & Xu's CNN& \textbf{0.898} &  0.824  & 0.805 &  0.849 &0.841 \\
   & Tuned SVM& 0.878 & \textbf{ 0.841}  &\textbf{ 0.969} &  \textbf{0.909} &\textbf{0.899} \\\hline
%   \multirow{2}{*}{Accuracy} &  Our SVM&0.550  &  0.402 & 0.970 & 0.645 &0.667 \\
 %  & Xu's SVM& - &  -  &- &  - &0.669 \\ \hline
 \end{tabular}}
\label{tab:RQ2}
\end{table}

\textbf{RQ2: Is tuning SVM comparable with Xu et al.'s deep learning method in terms of performance scores?}

To answer this question, we run the experiment designed as~\fig{workflow}, where DE is applied to 
find the optimal parameters for SVM based on the training and tuning data. Then the optimal tunings
applied on the SVM model and evaluate the built learner on testing data.

\tab{RQ2} says that the answer to RQ2 is yes and for some cases, tuning SVM is much better than
CNN method. In \tab{RQ2},  Xu et al's baseline, Xu et al's CNN method and Tuned SVM are
evaluated by all metics, where the highest score for each type of KU pair
relatedness is marked in bold.  Without tuning, Xu et al's CNN method outperform
the baseline SVM for all classes across all metics. However, after tuning SVM, the deep learning
method has a slightly better performance only on {\it duplicate} class for {\it precsion},{\it recall} and {\it F1-score} and 
 a higher {\it recall} on {\it indirect link} class. Other than that, it does not have any advantage in terms of evaluation
 metrics.
 
 Based on the comparison in \tab{RQ2}, we can see that: (1) parameter tuning improve baseline method;
 (2) With the optimal tunings, the traditional machine learning method, SVM, if not better, is at least comparable 
 with deep learning method, CNN, in terms of evaluation metrics.  \wei{Note that we are not saying tuning SVM is significantly
 better than deep learning method since it can't be concluded from this experiment without further statistical analysis.}
 Due the lack of distributions of CNN evaluation metrics, comparing a single value with our 10 repeats run results doesn't
 make sense. Therefore, this question will be investigated in the future work.
 
 
 \begin{lesson}
 Tuning help improve the performance of the  SVM baseline method for knowledge units relatedness prediction.
 After tuning, SVM has higher performance scores than CNN method on 3 classes of relatedness.
 The deep learning method, CNN, if not worse, is just as good as tuning baseline method.
 \end{lesson}
 

\textbf{ RQ3: Is tuning SVM comparable with Xu et al.'s deep learning method in terms of runtime?}
 
 To compare the runtime of two learning methods, it obviously should be conducted under the
 same hardware settings. Since we adopt the CNN evaluation metrics from ~\cite{xu2016predicting},
 we can't run on our tuning SVM experiment under the exactly same system settings. However, our experiment
 is run under an system which is inferior to the one used in \cite{xu2016predicting}. Therefore, such comparison
 still can provide some insights. 
 
 
 \begin{table}[!htp]
\centering
\caption{Comparison of Runtime and System   }
\resizebox{0.48\textwidth}{!}{
\begin{tabular} {@{}l  l l l l @{}}
\hline
Methods&OS&CPU&RAM&Runtime \\ \hline
Tuning SVM & Mac10.12 & Intel Core i5 2.7 GHz & 8GB & 10 minutes \\
CNN& Windows7 &Intel Core i7 2.5 GHz & 16GB &14 hours\\
\hline
\end{tabular}
}
\label{tab:runtime}
\end{table}
 
Deep learning method is  powerful to explore the data feature in depth.
However, the main disadvantage of it is that requires a lot of computational resources.
 According to Xu et al, the CNN method takes 14 hours to train the model
to get a relative low loss convergence. However, running  SVM with parameter tuning
by DE on a similar settings shown in \tab{runtime} takes 10 minutes to finish 
the whole experiment including parameter tuning, training model and prediction,
which is $80X$ faster than  Xu's deep learning method.

 
 
 
 





